%%%%%%%%%%%%%%%%%%%%%%%%%%%%%%%%%%%%%%%%%%%%%%%%%%%%%%%%%%%%%%%%%%%%%%%%%%%%%%%

\title{Rapport Projet}
\author{Paul Patault}
\date{Mars 2022}

\documentclass[a4paper,12pt]{article}

%%%%%%%%%%%%%%%%%%%%%%%%%%%%%%%%%%%%%%%%%%%%%%%%%%%%%%%%%%%%%%%%%%%%%%%%%%%%%%%
\usepackage[utf8]{inputenc}
\usepackage[T1]{fontenc}
\usepackage{tikz}
\usepackage{amsfonts}
\usepackage{pdfpages}
\usepackage{graphviz}
\usepackage{amsthm, amsmath, amssymb, extarrows}
\usetikzlibrary{decorations.text,calc,arrows.meta}
\usetikzlibrary{automata,positioning}% Set target color model to RGB
\usepackage{minted}

\usepackage[french]{babel}
\usepackage{caption}
\usepackage{float}
\usepackage{fancyhdr}
\usepackage[a4paper, left=3cm, right=3cm, bottom=2.5cm]{geometry}

\pagestyle{fancy}
\fancyhf{} \rhead{\small M1 MPRI @ Université Paris-Saclay}
\lhead{\small Paul Patault}
\rfoot{\thepage}
\pagenumbering{arabic}

%%%%%%%%%%%%%%%%%%%%%%%%%%%%%%%%%%%%%%%%%%%%%%%%%%%%%%%%%%%%%%%%%%%%%%%%%%%%%%%

\newcommand{\ocaml}[1]{\mintinline{ocaml}{#1}}

\renewcommand{\baselinestretch}{1.2}

\begin{document}

%%%%%%%%%%%%%%%%%%%%%%%%%%%%%%%%%%%%%%%%%%%%%%%%%%%%%%%%%%%%%%%%%%%%%%%%%%%%%%%

\setlength{\headheight}{13.59999pt}
\addtolength{\topmargin}{-1.59999pt}


\begin{center}
  % \vspace*{.5cm}

  \textbf{\Large Rapport de projet}

  % \vspace*{.5cm}

  % Paul Patault \\
  {\ttfamily paul.patault@universite-paris-saclay.fr} \\

  % \vspace*{1cm}

\end{center}

%%%%%%%%%%%%%%%%%%%%%%%%%%%%%%%%%%%%%%%%%%%%%%%%%%%%%%%%%%%%%%%%%%%%%%%%%%%%%%

\subsection*{2. Validation top-down non-déterministe}
\subsubsection*{Question 1}

Un run d'un automate d'arbre $A = (Q, \delta, I, F, \Sigma)$
pour un arbre $t \in {\mathcal{T}}(\Sigma)$
est une fonction $r : dom(t) \to Q$
telle que $\forall p \in dom(t), (t(p), r(p), r(p1),r(p2)) \in \delta$.
Un run est dit acceptant si et seulement si $r (\epsilon) \in I$.

\subsubsection*{Question 2}

  \begin{minted}[baselinestretch=1.2,fontsize=\footnotesize]{ocaml}
    let rec validate_td a t p q =
        let lab = label p in
        if lab = '#' then
            true
        else
            let l = get_trans a lab q in
            List.fold (fun acc elt ->
                acc || validate_td a t (elt p) q
            ) false l
        (* first_child(p), next_sibling(p) *)
   \end{minted}

\subsubsection*{Question 3}

La complexité de l'expression

\centerline{
$\exists$ \ocaml{q} $\in I$ tel que \ocaml{validate_td a t eps q}
}

est $O(|\ocaml{a}|^{|\ocaml{t}|})$. En effet, l'algorithme
nous fait prendre $|\ocaml{a}|$ fois chaque arête de l'arbre \ocaml{t}.

%%%%%%%%%%%%%%%%%%%%%%%%%%%%%%%%%%%%%%%%%%%%%%%%%%%%%%%%%%%%%%%%%%%%%%%%%%%%%%

\subsection*{3. Validation bottom-up}
\subsubsection*{Question 1}
\subsubsection*{Question 2}
\subsubsection*{Question 3}

%%%%%%%%%%%%%%%%%%%%%%%%%%%%%%%%%%%%%%%%%%%%%%%%%%%%%%%%%%%%%%%%%%%%%%%%%%%%%%

\subsection*{4. Compilation}
\subsubsection*{Question 1}
\subsubsection*{Question 2}

\end{document}
