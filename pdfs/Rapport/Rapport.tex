%%%%%%%%%%%%%%%%%%%%%%%%%%%%%%%%%%%%%%%%%%%%%%%%%%%%%%%%%%%%%%%%%%%%%%%%%%%%%%%

\title{Rapport Projet}
\author{Paul Patault}
\date{Mars 2022}

\documentclass[twoside,12pt]{article}
\usepackage[code, figs]{ppbase}

\begin{document}

%%%%%%%%%%%%%%%%%%%%%%%%%%%%%%%%%%%%%%%%%%%%%%%%%%%%%%%%%%%%%%%%%%%%%%%%%%%%%%%

\thispagestyle{empty}

\vspace*{1.3cm}
\begin{center}
    \textbf{\LARGE Automates et applications} \\
    \bigskip\bigskip
    {\large Paul Patault} \\ % \daggerfootnote{\texttt{paul.patault@universite-paris-saclay.fr}} \\
    \bigskip
    \today
    \bigskip
\end{center}

\begin{abstract}
    Projet du cours \textit{Automates et Applications}. Utilisation de la structure
    de donnée d'automates d'arbres pour vérifier la bonne formation de fichier type
    XML. Ces automates d'arbres seront générés automatiquement à partir d'un fichier
    \textit{à la} DTD donné en entrée du programme. Celui-ci sera donc parsé puis
    compilé dans notre type.
\end{abstract}

%%%%%%%%%%%%%%%%%%%%%%%%%%%%%%%%%%%%%%%%%%%%%%%%%%%%%%%%%%%%%%%%%%%%%%%%%%%%%%%

\section{Validation des documents}
\textit{Section sans questions. Nous profitons cependant de cet espace pour faire quelques
remarques.}

\subsection{Syntaxe}

\subsubsection{Fichiers DTD}

L'extension \texttt{dtd} est utilisé comme abus de langage mais clarifie les choses.
La syntaxe est simple, un type commence par le mot clé \og\texttt{type}\fg{} suivi
par le nom donné à ce dernier, ensuite un signe \og\texttt{=}\fg{} puis le nom de
la balise et enfin entre \og\texttt{[]}\fg{} se trouve une expression régulière
(où \og\texttt{*}\fg{} est le symbole classique, \og\texttt{|}\fg{} indique
l'alternative et \og\texttt{+}\fg{} la concaténation).
Voici un fichier \texttt{.dtd} qui sera accepté par le parser:
  \begin{minted}[baselinestretch=1.2,fontsize=\small, escapeinside=\/, mathescape=true]{xml}
            type t  = HD[ x* ]
            type x  = P[ y+z* ]
            type y  = B[ (z+z)|x* ]
            type z  = C[]
  \end{minted}

\subsubsection{Fichiers XML}

Nous utilisons la bibliothèque \texttt{Xml-light} pour le parsing.
Voici un fichier \texttt{.xml} d'exemple qui sera accepté par le parser (remarque :
ce fichier est bien typé par rapport au fichier \texttt{.dtd} proposé en exemple):
  \begin{minted}[baselinestretch=1.2,fontsize=\small, escapeinside=\/, mathescape=true]{xml}
  <HD>
      <P>
         <B>
            <C></C>
            <C></C>
         </B>
      </P>
      <P>
         <B></B>
         <C></C>
         <C></C>
      </P>
  </HD>
  \end{minted}

\subsection{À propos du sujet, et de l'implémentation}

Je n'ai pas réussi à faire le sujet complet.
Le pseudo-code de la section 4 n'est pas exact, et
le code réel associé ne mène pas encore à un résultat concluant.
Je n'ai pas complètement trouvé comment concevoir l'algorithme de
compilation d'un fichier de type \texttt{DTD} vers un automate
d'arbre binaire \footnotemark. Le point qui m'a en particulier posé
problème est la compilation des transitions. En effet, je ne trouve
pas de façon générale pour réaliser cette transformation. Je crois
que mon problème est précisément situé au niveau des états à nommer
avant la vérification de cohérence. Ayant un code pour gérer les
expressions régulières, il faudrait \og simplement \fg{} transporter
en plus l'état courant de l'automate classique en plus.

\footnotetext{Je précise bien binaire car il s'agit selon moi (je peux me tromper)
    d'une difficulté supplémentaire imposée dans ce projet, je pense qu'avec une
    structure d'arbre classique la compilation des transitions aurait été plus simple.}

%%%%%%%%%%%%%%%%%%%%%%%%%%%%%%%%%%%%%%%%%%%%%%%%%%%%%%%%%%%%%%%%%%%%%%%%%%%%%%%

\section{Validation top-down non-déterministe}
\subsection{Question}

Un run d'un automate d'arbre $A = (Q, \delta, I, F, \Sigma)$
pour un arbre $t \in {\mathcal{T}}(\Sigma)$
est une fonction $r : dom(t) \to Q$
telle que $\forall p \in dom(t), (t(p), r(p), r(p1),r(p2)) \in \delta$.
Un run est dit acceptant si et seulement si $r (\epsilon) \in I$.

\subsection{Question}

\begin{algorithm}
    \bigskip
  \begin{minted}[baselinestretch=1.2,fontsize=\small]{ocaml}
    let rec validate_td a t p q =
      if label (t p) = '#' then is_final q
      else
        let l =
          List.filter
            (fun (l, q', _, _) -> l = label (t p) && q = q)
            a.delta
        in
        List.fold
          (fun acc q' ->
            acc
            || validate_td a t (p @ [ first_child (t p) ]) q'
               && validate_td a t (p @ [ next_sibling (t p) ]) q')
          false l
   \end{minted}
   \caption{\small Pseudo-code \textit{à la} Caml pour la fonction \texttt{validate\_td}}
\end{algorithm}

\subsection{Question}

La complexité de l'expression
$\exists$ \ocaml{q} $\in I$ tel que \ocaml{validate_td a t eps q}
est $O(|\ocaml{a}|^{|\ocaml{t}|})$, où $|\ocaml{a}|$ est le nombre de
transitions de l'automate \ocaml{a}. En effet, l'algorithme
nous fait prendre au pire $|\ocaml{a}|$ fois chaque arête de l'arbre \ocaml{t}.

%%%%%%%%%%%%%%%%%%%%%%%%%%%%%%%%%%%%%%%%%%%%%%%%%%%%%%%%%%%%%%%%%%%%%%%%%%%%%%

\section{Validation bottom-up}

\subsection{Question}
\begin{algorithm}
    \bigskip
  \begin{minted}[baselinestretch=1.2,fontsize=\small]{ocaml}
    let rec validate_bu a t p =
      let lab = label (t p) in
      if lab = '#' then
        List.filter
          (fun (lab', _, l', r') -> '#',[],[] = lab',l',r' )
          a.delta
    else
      let left = validate_bu a t (p@[first_child (t p)]) in
      let right = validate_bu a t (p@[next_sibling (t p)]) in
      let res = ref [] in
      List.iter (fun r -> List.iter (fun l ->
          let trans = lab, l, r in
          let possible_states =
            List.filter
              (fun (lab', _, l', r') -> lab,l,r = lab',l',r' )
              a.delta
          in
          res <- possible_states :: !res;
        ) left) right
        res
   \end{minted}
   \caption{\small Pseudo-code \textit{à la} Caml pour la fonction \texttt{validate\_bu}}
\end{algorithm}


\subsection{Question}

La complexité de l'expression (\ocaml{validate_bu a t eps}) $\cap\ I$ est $O(|\ocaml{t}|)$.

\subsection{Question} TODO

%%%%%%%%%%%%%%%%%%%%%%%%%%%%%%%%%%%%%%%%%%%%%%%%%%%%%%%%%%%%%%%%%%%%%%%%%%%%%%

\section{Compilation}
\subsection{Question}
En quelques mots, nous commençons par transformer l'arbre $n$-aire parsé
par la librairie \texttt{Xml-light} en un arbre binaire (le code est situé
dans le fichier \texttt{src/tree.ml}) avec un algorithme simple s'appuyant
sur l'isomorphisme de ces deux structures présenté en fin de cours 3. Ainsi
les fils directs de chaque n\oe{}ud deviennent les fils droits du fils gauche
de ce n\oe{}ud et récursivement.

\begin{algorithm}
    \bigskip
\begin{minted}[baselinestretch=1.2,fontsize=\small, escapeinside=\/, mathescape=true]{ocaml}
    let n2bin input_n_tree =
      let rec aux = function
        | [] -> Leaf
        | node :: sibling -> Node (aux node, aux sibling)
      in
      match input_n_tree with
      | rac, childs -> Node (rac, aux childs, Leaf)
\end{minted}
   \caption{\small Pseudo-code \textit{à la} Caml pour la fonction de compilation des arbres
   n-aires vers des arbres binaires}
\end{algorithm}

Pour la compilation des expressions régulières, nous utilisons la construction
de Berry-Sethi \footnote{Elle semble être la même que celle de Glushkov)}.
Celle-ci est efficace, et n'a pas le défaut de Thompson en générant des transitions
epsilon qu'il aurait par la suite fallut supprimer.
Le code se trouve dans le fichier \texttt{src/regautomata.ml} et parle de lui-même, nous n'allons
donc pas le présenter plus en détails.

Enfin pour la compilation des types \textit{à la} DTD en eux même,
la solution n'est pas encore très claire. Arrivé seul à la même idée
que celle proposé dans le message sur e-campus
\footnote{Message que je n'ai pu découvrir que ce dimanche, n'ayant
pas eu de notification relative aux messages, probablement dû à leur
caractère \og semi-privés \fg{} sur e-campus.},
l'avancement s'est fait à tâtons et aucune conclusion n'est envisageable dans l'immédiat.

L'implémentation est donc une tentative d'application d'une idée vague relativement similaire à
celle discutée. La fonction de compilation d'un fichier de type est située dans
\texttt{src/compiler.ml} et il s'agit de la fonction \texttt{compile\_typ}.
L'idée est dans un premier temps de remplir un table de hashage contenant les types, pour
regrouper les possibles multiples définitions d'un même type. Ensuite, en itérant sur chacun
de ces derniers, l'automate de mot correspondant à l'expression régulière est fabriqué, et
nous tentons d'encoder ces états dans les transitions de notre automate d'arbre en construction.
Cependant, il doit y avoir une erreur à ce niveau dans le code, mais la localisation précise
ce bogue pose encore problème.



\subsection{Question}
Ayant une complexité en $O(n^2)$ pour la construction de Glushkov, et en entrée $n$ définitions,
nous avons une complexité en temps en $O(n^3)$\footnotemark.
Pour la complexité en espace, l'automate d'arbre produit correspondant
au type donné en entrée serait de taille $O(TODO)$.

\footnotetext{En supposant que la taille des expressions régulières
est du même ordre de grandeur que le nombre de types.}

%%%%%%%%%%%%%%%%%%%%%%%%%%%%%%%%%%%%%%%%%%%%%%%%%%%%%%%%%%%%%%%%%%%%%%%%%%%%%%

\end{document}
